\subsection*{Learning Objectives}
	Students need to be able to\ldots
	\begin{itemize}
		\item Switch between phase portraits and component graphs.
		\item Recognize the difference between solutions and curves in phase space.
		\item Deduce properties of equilibrium solutions from phase portraits.
	\end{itemize}

\subsection*{Context}
	We have analyzed systems of ODEs in class and in tutorial using phase portraits. We are getting ready to
	study systems of ODEs and equilibrium solutions to system written in matrix form. This requires a very good
	understanding of phase portraits and how they relate to solutions to systems.

\subsection*{What to Do}
	Introduce the learning objectives for the day's tutorial. Explain that phase portraits/graphs in phase space are an important
	tool in the study of differential equations and we want to better understand how graphs in phase space
	relate to solutions to systems of ODEs.

	Start by asking students to recall what we mean when we say ``phase portrait'' and ``component space'' for a system of ODEs
	$A'(t)=\ldots$ and $B'(t)=\ldots$. \emph{Do not give a lecture on this}, but you may have a short discussion ($< 5$ min)
	on this distinction.
	
	After most groups have finished \#1, go over it as a class. Doing so will ensure everyone has a baseline understanding
	of the distinction between phase space and component space.

	Continue as usual, walking around the room and asking
		questions while letting students work on the next problem and gathering them together
		for discussion when most groups have finished.

		7 minutes before class ends, pick a suitable problem to do as a wrap-up. Most likely, \#3 will be a good choice.



	
\subsection*{Notes}
	\begin{enumerate}
		\item Hopefully this problem won't be hard for them. All the vertices line up, after all. If they are struggling,
		it is worth spending a lot of time on this question, since they cannot use phase portraits unless they understand this.

		\item Students will struggle getting a curve for this question. If they are struggling, have them approximate
		the curve with a polygon and try to find component graphs for the polygon first. Then they can ``smooth it out''
		to get the curve.
		
		Part (b) will be especially hard. Some will struggle to even understand what the question is asking.

		\item This question is a good wrap-up question. It should be quicker than all the other questions.

		\item This is a question for groups who have moved more quickly. It will be very hard for the students and requires a
		mastery of pre-calculus concepts. Students may need some prodding on part (b) about stretching the domain and what
		types of functions will give them an appropriate stretch/compression.

		%\item This is a question for groups who have moved more quickly. Students may have forgotten what an equilibrium
		%solution is. Ask them to check their notes for the definition.

		%Part (c) introduces a brand new definition. Assure students that they have not learned this definition in class
		%and that's okay!

	\end{enumerate}