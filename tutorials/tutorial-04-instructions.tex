\subsection*{Learning Objectives}
	Students need to be able to\ldots
	\begin{itemize}
		\item Recall and use Linear Algebra concepts from MAT223
		\item Use existing solutions to find other solutions to linear ODEs
	\end{itemize}

\subsection*{Context}
	We have just started applying linear algebra concepts to MAT244. We have written systems of first-order ODEs
	in matrix form and are exploring how eigenvectors of that matrix relate to eigen solutions and general solutions.
	However, students are very fuzzy on their Linear Algebra. About 1/3rd of the students are currently taking MAT224,
	but many of the others haven't seen Linear Algebra in at least 6 months. And, their foundation may have been shaky to begin with.

\subsection*{What to Do}
	Introduce the learning objectives for the day's tutorial. Explain that linear algebra concepts are
	very useful in analyzing and solving differential equations. Also explain that while you can look up
	any concepts you've forgotten, it is much better for memory to struggle to remember on your own \emph{before}
	you look up concepts you may have forgotten.

	Have students pair up and start with a \textbf{Warmup:}
	\begin{itemize}
		\item Write down the definition of the \emph{null space} of a linear transformation.
	\end{itemize}

		Many will struggle. After 3 minutes, discuss the definition as a whole class (it is necessary to make progress on \#1).

		After the warmup, have the students work in groups on the remaining problems. Circulate and ask/answer questions as usual.

		7 minutes before the end of class, pick a problem that most students have started working on
		to do as a wrap-up.

\subsection*{Notes}
	\begin{enumerate}
			\item In part (a) students may need a hint. Ask them: ``What does $\vec u$ correspond to in the expression $\Null(M)+\{\vec p\}$?''
			Part (c) we don't have enough information to answer definitively (the vector $\vec b$ could be the zero vector, after all), but we can
			answer in the ``general'' case.

			\item Part (a) should be straight forward. Part (b) will make many students confused and uncomfortable. We're asking them to take
			a leap of faith. The wording of the problem is to ``guess'' a solution, not to ``find'' a solution. Remind them that it is very quick
			to check if a guess is correct, so they can try lots of things.

			Part (d) can be answered by actually differentiating the expressions or with a hand-wavy answer like 1(c).

			\item This question steps the abstraction up and is much harder. It is there fore the students who have a solid linear
			algebra background. If you see students working on this question, make sure they can explain questions \#1 and \#2. If they
			cannot, send them back to those questions.
	\end{enumerate}
