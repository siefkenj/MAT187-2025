\begin{objectives}
	In this tutorial, you will explore separable and first-order linear ODEs.
\end{objectives}

\subsection*{Problems}
\begin{enumerate}
    \begin{minipage}[t]{\linewidth-7cm}
    \item Consider the ODE $y'+y=t$. Is there more than one integrating factor that can be used to solve it?

    \item Consider the ODE
\[
    \frac{\mathrm dy}{\mathrm d t} = f(y) = (y-1)^2 (y-2) (y-3)
\]
The phase plot of this ODE is given on the right.

\textit{You can answer this question based on the equation, the phase plot, or both!}
\begin{enumerate}
\item Find and classify the equilibrium points of the differential equation. Make sure to justify.
    \vspace{3mm}

\item Consider again the ODE $\frac{\mathrm d y}{\mathrm d t} = f(y) = (y-1)^2 (y-2) (y-3)$ from the previous question. If $y(t)$ is a solution and $y(0.5)=2.8$, what is $\displaystyle \lim_{t\to\infty} y(t)$? Justify your choice.

\end{enumerate}
\end{minipage}\hfill
\begin{minipage}[t]{6cm}
	\vfil
	\vspace{-6mm}
	\flushright
	\begin{tikzpicture}[line width=1]
	
		\draw [black!20] (-0.5,-1) grid [step=0.5] (4,3);
		\begin{scope}
		\clip (0,-1) rectangle (4,3);
		\draw [smooth,samples=100,domain=0:3.5,tolOrange] plot ({\x},{(\x-1)*(\x-1)*(\x-2)*(\x-3)});
		\end{scope}
	\draw [->] (-0.5,0) -- (4,0) node [right] {$y$};
	\draw [->] (0,-1) -- (0,3) node [above] {$f(y)$};
	\draw (1,0.1) -- (1,-0.1) node [below] {1};
%	\draw (0.1,1) -- (-0.1,1) node [left] {1};
\end{tikzpicture}

\end{minipage}
\item Solve the following first order ODEs using separation of variables, and determine over what interval the solution is defined.

\begin{enumerate}
    \item $\frac{\mathrm d y}{\mathrm d x} = 6y^2x$, $y(1) = \frac{1}{25}$

    \item $\frac{\mathrm d r}{\mathrm d \theta} = \frac{r^2}{\theta}, r(1) = 2 $  

    \item $\frac{\mathrm d y}{\mathrm d t} = e^{-y}(2x-4), y(5)=0$ 
\end{enumerate}

\item 
A hungry tiger is hiding in a bush at a position $(L, 0)$, intently watching a gazelle grazing at the origin. To catch the gazelle, the tiger's strategy is to run at a constant speed and towards the gazelle at all times. The gazelle's strategy is to run in the positive $y$ direction at the same speed as the tiger. The equation of motion of the tiger is represented by the differential equation:
	
	\[
	    \sqrt{1+\left(\frac{dy}{dx}\right)^2} = x\frac{d^2y}{dx^2}
	\]
\begin{enumerate}
	    \item Apply the substitution $z=\frac{dy}{dx}$. You should now be able to find $z(x)$ and then $y(x)$, which is the tiger's trajectory.
	    
	    \textit{Hint: $\int\sec{\theta}d\theta = \ln|\sec{\theta} + \tan{\theta}| + C$.}
    \item Will the tiger ever catch the gazelle? Justify your answer.

\end{enumerate}
\end{enumerate}