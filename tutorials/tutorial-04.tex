\documentclass[red]{tutorial}
\usepackage[no-math]{fontspec}
\usepackage{xpatch}
	\renewcommand{\ttdefault}{ul9}
	\xpatchcmd{\ttfamily}{\selectfont}{\fontencoding{T1}\selectfont}{}{}
	\DeclareTextCommand{\nobreakspace}{T1}{\leavevmode\nobreak\ }
\usepackage{polyglossia} % English please
	\setdefaultlanguage[variant=us]{english}
%\usepackage[charter,cal=cmcal]{mathdesign} %different font
%\usepackage{avant}
\usepackage{microtype} % Less badboxes


\usepackage[charter,cal=cmcal]{mathdesign} %different font
%\usepackage{euler}
 
\usepackage{tikz}
\usepackage{pgfplots}
\usetikzlibrary{arrows.meta}

\usepackage{blindtext}
\usepackage{calc, ifthen, xparse, xspace}
\usepackage{makeidx}
\usepackage[hidelinks, urlcolor=blue]{hyperref}   % Internal hyperlinks
\usepackage{mathtools} % replaces amsmath
\usepackage{bbm} %lower case blackboard font
\usepackage{amsthm, bm}
\usepackage{thmtools} % be able to repeat a theorem
\usepackage{thm-restate}
\usepackage{graphicx}
\usepackage{xcolor}
\usepackage{multicol}
\usepackage{fnpct} % fancy footnote spacing

 
\newcommand{\xh}{{{\mathbf e}_1}}
\newcommand{\yh}{{{\mathbf e}_2}}
\newcommand{\zh}{{{\mathbf e}_3}}
\newcommand{\R}{\mathbb{R}}
\newcommand{\Z}{\mathbb{Z}}
\newcommand{\N}{\mathbb{N}}
\newcommand{\proj}{\mathrm{proj}}
\newcommand{\Proj}{\mathrm{proj}}
\newcommand{\Perp}{\mathrm{perp}}
\newcommand{\Span}{\mathrm{span}\,}
\newcommand{\Img}{\mathrm{img}\,}
\newcommand{\Null}{\mathrm{null}\,}
\newcommand{\Range}{\mathrm{range}\,}
\newcommand{\rref}{\mathrm{rref}}
\newcommand{\Rank}{\mathrm{rank}}
\newcommand{\nnul}{\mathrm{nullity}}
\newcommand{\mat}[1]{\begin{bmatrix}#1\end{bmatrix}}
\renewcommand{\d}{\mathrm{d}}
\newcommand{\Id}{\operatorname{id}}


\theoremstyle{definition}
\newtheorem{example}{Example}[section]
\newtheorem{defn}{Definition}[section]

%\theoremstyle{theorem}
\newtheorem{thm}{Theorem}[section]

\pgfkeys{/tutorial,
	name={Tutorial 4},
	author={},
	course={MAT 187},
	date={},
	term={},
	title={Integration Techniques}
	}

\begin{document}
	\begin{tutorial}
			\begin{objectives}
        In this tutorial you practice manipulating integrals.
	\end{objectives}

		\vspace{-.5em}
		\subsection*{Problems}
		\vspace{-.5em}


\begin{enumerate}
    \item 
    \begin{enumerate}
        \item Compute $\displaystyle\int\log_b x\,dx$ where $b$ is any positive number.
        \item You are given that $\displaystyle\int_1^2 xf'(2x)\,dx=8$ and a table of values of $f$. Find $\displaystyle\int_1^2 f(2x)\,dx$.
    	\begin{align*}
    	    \hfill\begin{tabular}{c||c|c|c|c|c|c|c}
            $x$&$0$&$1$&$2$&$3$&$4$&$5$&$6$\\
            \hline
            $f(x)$&$2$&$-4$&$10$&$6$&$-2$&$8$&$-12$
            \end{tabular}\hfill\null
    	\end{align*}
    \end{enumerate}

    \item
    \begin{enumerate}
        \item Find the integral $\displaystyle\int\frac{dx}{x^2-2x+10}$.
        \begin{minipage}{\linewidth-5cm}
        \item The upper half of an ellipse centered at the origin with axes $a$ and $b$ is described by $y = \frac{b}{a}\sqrt{a^2-x^2}$ (see figure). Find the area of the ellipse in terms of $a$ and $b$.
        \end{minipage}\hfill
        \begin{minipage}{4cm}
    	\flushright
    	\begin{tikzpicture}[scale=0.45,line width=1]
    		\draw [tolTeal] (0,0) ellipse (3 and 2);
    			\draw[->] (-4,0) -- (4.2,0) node [below] {$x$};
    		\draw[->] (0,-2.5) -- (0,3.2) node [left] {$y$};
    		\node at (3,0) [below right,tolTeal] {$a$};
    		\node at (0,2) [above left,tolTeal] {$b$};
    	\end{tikzpicture}
    	\end{minipage}
    \end{enumerate}

    \item 
    \begin{enumerate}
        \item Evaluate $\int \frac{1}{x^2 - 9}\,dx$ for $x > 3$ using trigonometric substitution.
        \begin{minipage}{\linewidth-5cm}
        \item 
        A ball with radius 1 is placed inside a cone that has a vertical slope of 1. Determine the cross sectional area of the region underneath the ball but within the cone (grey in the figure).
        
        \textit{Hint: Use the fact that the ball must be tangent to the cone.}
        \end{minipage}\hfill
        \begin{minipage}{4cm}
	    \flushright
		\begin{tikzpicture}[scale=0.75,line width=1]
		\fill[gray] (0,0) -- ({1/sqrt(2)},{1/sqrt(2)}) arc (-45:-135:1) --cycle;
		\draw [tolMagenta] (0,{sqrt(2)}) circle (1);
		\draw [tolTeal] (-2,2) -- (0,0) -- (2,2);
		\draw[->] (-2,0) -- (2,0) node [below] {$x$};
		\draw[->] (0,-0.5) -- (0,3) node [left] {$y$};
	\end{tikzpicture}
	\end{minipage}
    \end{enumerate}

    \item The Gamma function is defined by the formula 

    \[\Gamma(x) = \int_0^\infty e^{-t}t^{x-1}dt\]
    
    It's an extension of the factorial function that works for non-integer values. If you ever graph $x!$ in desmos, this is what's being graphed!
    \begin{enumerate}
    \item Use the comparison test to show that the integral converges when $x>0$.
    
    \item Show by integration by parts that:

    \[\Gamma(x) = x\Gamma(x-1)\]
    \item Show that $\Gamma(1) = 1$ and $\Gamma(n) = n!$.
    \end{enumerate}    
\end{enumerate}


















	\end{tutorial}

	\begin{solutions}
		\begin{enumerate}
		\item \begin{enumerate}
			\item We know that $\vec 0\in \Null(M)$. We also know that $\vec u-\vec v=\mat{3\\-1}\in \Null(M)$. Since $\Null(M)$
			is a subspace, we also have that $2\mat{3\\-1}\in\Null(M)$, along with many others.
			\item We can add a vector in the null space to any existing solution. For example, $\vec u+\mat{3\\-1}$, $\vec u+\mat{6\\-2}$, 
			and $\vec u+\mat{9\\-3}$ are all solutions.
			\item No. Unless $M\vec 0=\vec b$, we do not expect the solution set to the a subspace. If the solution set is not a subspace
			it is unlikely that summing two random solutions will result in another solution.
		\end{enumerate}
		
		\item \begin{enumerate}
			\item Differentiating, we see $u''+u=t^2-2$ and $v''+v=t^2-2$.
			\item By computing $u(t)-v(t)=-2\sin t$, we can guess that $u-2\sin$, $u-4\sin$, and $u-6\sin$ are all solutions. Computing,
			we verify that they indeed are.
			\item The transformation can be described in words by ``Take the second derivative of the function and then add one copy of the original function''.
			\item Similarly to the previous problem, we do not expect the solution set to be a subspace. However, with the transformation described in
			part (c), we can realize $t^2-2+K\sin t$ as a particular solution, $t^2-2$, plus a multiple of something in the null space, $\sin t$.
		\end{enumerate}
		
		\item \begin{enumerate}
			\item Let $f,g\in \mathcal C$ and let $k$ be a scalar. Then 
			\[\Id(f+g)=f+g=\Id(f)+\Id(g)\]
			similarly,
			\[
				\Id(kf)=kf=k\Id(f)
			\]
				and so $\Id$ is linear.

			For $D^2$, we compute
			\[
				D^2(f+g)=(f+g)''=f''+g''=D^2(f)+D^2(g)
			\]
			 and 
			 \[
			 	D^2(kf)=(kf)''=k(f'')=kD^2(f)
			 \]
				and so $D^2$ is linear.

			\item The sum of linear transformation is linear, so $T$ is linear.
			\item Computing $T(\sin) = -\sin+\sin=0$ and $T(\cos) = -\cos+\cos=0$ and so $\sin,\cos\in \Null(T)$.
			\item First note that $\{\sin,\cos\}$ is a linearly independent set. Therefore $\Null(T)=\Span\{\sin,\cos\}$.

			Now, notice that $t^2-2$ is a particular solution to $T(f)=t^2$. Because the equation $T(f)=t^2$ is a linear equation,
			we can write the complete solution as $\text{particular}+\Null(T)$ and so the complete solution to $T(f)=t^2$ is
			\[
				\{t^2-2\}+\Span\{\sin t,\cos t\}.
			\]
			Finally, notice that solutions to $T(f)=t^2$ are the same thing as solutions to $y''+y=t^2$.
		\end{enumerate}
\end{enumerate}
	

	
	\end{solutions}
	\begin{instructions}
		\subsection*{Learning Objectives}
	Students need to be able to\ldots
	\begin{itemize}
		\item Recall and use Linear Algebra concepts from MAT223
		\item Use existing solutions to find other solutions to linear ODEs
	\end{itemize}

\subsection*{Context}
	We have just started applying linear algebra concepts to MAT244. We have written systems of first-order ODEs
	in matrix form and are exploring how eigenvectors of that matrix relate to eigen solutions and general solutions.
	However, students are very fuzzy on their Linear Algebra. About 1/3rd of the students are currently taking MAT224,
	but many of the others haven't seen Linear Algebra in at least 6 months. And, their foundation may have been shaky to begin with.

\subsection*{What to Do}
	Introduce the learning objectives for the day's tutorial. Explain that linear algebra concepts are
	very useful in analyzing and solving differential equations. Also explain that while you can look up
	any concepts you've forgotten, it is much better for memory to struggle to remember on your own \emph{before}
	you look up concepts you may have forgotten.

	Have students pair up and start with a \textbf{Warmup:}
	\begin{itemize}
		\item Write down the definition of the \emph{null space} of a linear transformation.
	\end{itemize}

		Many will struggle. After 3 minutes, discuss the definition as a whole class (it is necessary to make progress on \#1).

		After the warmup, have the students work in groups on the remaining problems. Circulate and ask/answer questions as usual.

		7 minutes before the end of class, pick a problem that most students have started working on
		to do as a wrap-up.

\subsection*{Notes}
	\begin{enumerate}
			\item In part (a) students may need a hint. Ask them: ``What does $\vec u$ correspond to in the expression $\Null(M)+\{\vec p\}$?''
			Part (c) we don't have enough information to answer definitively (the vector $\vec b$ could be the zero vector, after all), but we can
			answer in the ``general'' case.

			\item Part (a) should be straight forward. Part (b) will make many students confused and uncomfortable. We're asking them to take
			a leap of faith. The wording of the problem is to ``guess'' a solution, not to ``find'' a solution. Remind them that it is very quick
			to check if a guess is correct, so they can try lots of things.

			Part (d) can be answered by actually differentiating the expressions or with a hand-wavy answer like 1(c).

			\item This question steps the abstraction up and is much harder. It is there fore the students who have a solid linear
			algebra background. If you see students working on this question, make sure they can explain questions \#1 and \#2. If they
			cannot, send them back to those questions.
	\end{enumerate}

	\end{instructions}

\end{document}
