\begin{enumerate}
	\item
	      \begin{enumerate}
		      \item
		            \textbf{Equation (A)}: Stable and attracting.

		            \textbf{Equation (B)}: Stable and attracting.

		            \textbf{Equation (C)}: Unstable and repelling.

				\item Equation (A) has an affine approximation $y'=-y$. Equations (B) and (C) 
					have same affine approximation: $y'=0$.
		      \item Every affine approximation has a stable equilibrium solutions at $0$. The affine approximation for Equation (A) is
			  	also attracting.
		      \item Affine approximations are based off of the first derivative. For Equations (B) and (C), the first
		            derivative at the origin is zero, so it fails to differentiate between cases.
		      \item Given an autonomous equation $y'=f(y)$ with an equilibrium at $y=k$, the equilibrium
		            is attracting if $f'(k) < 0$, repelling if $f'(k) > 0$ and more investigation is needed if $f'(k) = 0$.
	      \end{enumerate}

	\item
	      \begin{enumerate}
		      \item $\vec r\,' = \mat{0&-1\\1&0}\vec r$.
		      \item The equilibrium for the affine approximation is stable (not repelling nor attracting).
		      \item No. Similar to Question 1, there are cubics that the first derivative fails to pick up on. These
		            cubics surely cause problems!
		      \item It looks like solutions circle about the origin, but it is hard to tell much beyond that.
		      \item Based on numerical simulations using Euler's method with $\Delta <0.01$, it appears that solutions
		            very slowly circle inwards.
		      %\item If we consider the differential equations $\vec r\,' =\mat{-y\\x}$ and $\vec r\,'=\mat{-x^3\\-y^3}$
		      %      in isolation, the first one has periodic solutions that circle about the origin and the second one
		      %      has solutions that head straight towards the origin. Equation (D) is a sum of these two equations, so
		      %      it makes sense that solutions circle but also tend towards the origin.

		      %      We can quantify how much solutions move towards the origin. We know $\vec r\,'(x,y)$ is a tangent
		      %      vector to a solution curve at the point $(x,y)$ and that the vector $(x,y)$ points radially out
		      %      from the equilibrium solution to the point $(x,y)$. Thus, computing
		      %      \[
			  %          \vec r\,'\cdot \mat{x\\y}=\mat{-y-x^3\\x-y^3}\cdot \mat{x\\y}=-(x^4+y^4)
		      %      \]
		      %      we see that the angle between $\vec r\,'$ and $(x,y)$ is always greater than $90^\circ$. Thus, solution
		      %      curves tend slightly towards the origin, making the equilibrium attracting.

		      \item If the real part of all eigenvalues is non-zero we can use the eigenvalues to classify the equilibrium
		            as follows. If all real parts are positive, the equilibrium is repelling and unstable. If all real parts are negative, the equilibrium
		            is attracting and stable. If there is a mix of positive and negative real parts, the equilibrium is unstable.

		            Otherwise, if at least one real part is positive, we know the equilibrium is unstable (but cannot determine whether it is repelling).
		            If at least one real part is positive and one real part is negative, we know the equilibrium is unstable
		            and not repelling.

		            In all other cases, we cannot conclude from the eigenvalues the nature of the equilibrium.
	      \end{enumerate}
\end{enumerate}