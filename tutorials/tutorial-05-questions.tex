\begin{objectives}
	In this tutorial you will practice evaluating improper integrals.
\end{objectives}

\vspace{-.5em}
\subsection*{Problems}
\vspace{-.5em}

%%%%%%%%%%%%%%%%%%%%%%%%%%

\begin{enumerate}
	\item 
	
\begin{enumerate}
    \item Find an anti-derivative of  $\frac{1}{x^2 - 9}$ using partial fraction decomposition.
    \item Find an anti-derivative of  $\frac{1}{9 + x^2}$ using trigonometric substitution.
    \item (\textbf{Challenge}) Find an anti-derivative of  $\frac{1}{9 + x^2}$ using partial fraction decomposition. Does your answer match the answer you got using trig substitution?

    \emph{Hint:} You may need to use complex numbers during your partial fraction decomposition step, but after simplifying, you should be left with a real answer.

    If a complex number $z=re^{i\theta}$, then $\ln z = \ln r + i\theta$.
\end{enumerate}

\item The magnitude of the force acting on a particle of charge $q_1$ that is $r$ meters away from another particle of charge $q_2$ is
   	\[
	  	F =k\frac{q_1 q_2}{r^2}
   	\]
    where $k = 8.99 \times 10^9$ Nm$^2$/C$^2$. Suppose both $q_1$ and $q_2$ are positive, which means the two particles are being pushed away from each other (``opposites attract, likes repel''). Use integrals to answer the following.
    \begin{enumerate}
        \item The electrons start out at a distance of $10$m.
        The charge of the electrons is $q_1=q_2=1.6\times 10^{-19}$.

        How much work is required to move the electrons so that they
        are only $1$m apart?
        
        \emph{Hint:} The work is done \textit{against} the force field and therefore is given by the integral of $-F(r)$ over distance from $r=a$ to $r=b$.%(work done against a force over a distance is $\int_a^b F(r) \,  dr$)
        \item Is it possible to move the electrons so that they are $0$ meters apart?
        \item If the electrons started very, very far away, would it be possible to bring them into a distance of $1$m from each other with a 
        finite amount of work?
    \end{enumerate}



\item The Gamma function is defined by the formula 

    \[
        \Gamma(x) = \int_0^\infty e^{-t}t^{x-1}\,\mathrm dt
    \]
    It's an extension of the factorial function that works for non-integer values. If you ever graph $x!$ in desmos, this is what's being graphed! In this question we'll show that the Gamma function extends the factorial function.
   \begin{enumerate}
        \item Graph $x!$ in Desmos.
    
        \item Let $n$ be an integer with $n\geq 2$. Use integration by parts to show that:

        \[
            \Gamma(n) = (n-1)\Gamma(n-1)
        \]

        \item Show that $\Gamma(1) = 1$

        \item Using the formula above, find a value for $\Gamma(4)$.
        

        
        \item With the same assumptions, can you find a formula for $\Gamma(n)$?
        \item Does your recursive formula for $\Gamma$ still hold
        when computing $\Gamma(x)$ when $x\in \R$ and $x>1$? Explain.
    \end{enumerate}    

\item (\textbf{Challenge}) In the last question, you showed that $\Gamma(n)=(n-1)\Gamma(n-1)$, which means if $\Gamma(1)=1$, the function $\Gamma(x)$ will have a finite value for integers $x\geq 1$.


The topic of this question is to prove that this integral converges for any $x>0$. To do this, we will break it into parts. 

\begin{enumerate}
    \item Consider the integrand $g_x(t)=e^{-t}t^{x-1}$.
    
    \begin{enumerate}
        \item For which values of $x$ does $g_x$ have a vertical asymptote? For which values of $x$ does $g_x$ have a horizontal asymptote?
        For which values of $x$ is $g_x$ a non-negative function?
        \item Show that if $t\geq 1$, then $g_{x_1}(t) \geq g_{x_2}(t)$ so
        long as $x_1\geq x_2\geq 1$.
    \end{enumerate}

    \item We know that $\Gamma(n)$ has a finite value, and so $\Gamma$ at any integer $n\geq 1$ is
     \emph{finite}. Using this fact, apply a comparison test to show that
      $\Gamma(x)<\infty$ when $x\geq1$.

      \emph{Hint:} You may need to do separate comparisons for $t<1$ and $t\geq 1$.
    
    \item Let's show that $\Gamma(x)$ converges for $0<x<1$. Split up $\Gamma(x)$ into the following two integrals:

    \[\Gamma(x) = \Gamma_1(x) + \Gamma_2(x) = \int_0^1 e^{-t}t^{x-1}\mathrm dt + \int_1^\infty e^{-t}t^{x-1}\mathrm dt\]

    If both $\Gamma_1$ and $\Gamma_2$ converge, then we're good!

    \begin{enumerate}
        \item Show that $e^{-t}t^{x-1} \leq t^{x-1}$ and then 
        use a comparison test to show that $\Gamma_1(x)$ converges. 
        \item Show that $e^{-t}t^{x-1} \leq e^{-t}$ and then use a comparison test to show $\Gamma_2(x)$ converges.
    \end{enumerate}

    
\end{enumerate}



\end{enumerate}

%%%%%%%%%%%%%%%%%%%%%%%%%%