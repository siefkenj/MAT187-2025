\begin{objectives}
	In this tutorial you will MAT 244's expectations around mathematical communication as well
	as what makes a good executive summary.
\end{objectives}

\vspace{-.5em}
\subsection*{Problems}
\vspace{-.5em}

%%%%%%%%%%%%%%%%%%%%%%%%%%

\begin{enumerate}
	\item 
	
\begin{enumerate}
    \item Evaluate $\int \frac{1}{x^2 - 9}\,dx$ for $x > 3$ using partial fractions.
    \item How many coefficients ($A,B,C,...$) are needed to set up the partial fraction decomposition of this?
	\[
		\frac{x^3-3x+8}{(x^2-1)^2(x^2+1)(x+\pi)^3(x^2+x+1)}
	\]
\end{enumerate}

\item Evaluate $\displaystyle\int \frac{1}{x^2 + a}\,dx$ where $a$ is a constant.
	
	\textit{Hint: Consider different possible values of $a$.}

\item The magnitude of the force acting on a particle of charge $q_1$ that is $r$ meters away from another particle of charge $q_2$ is
   	\[
	  	F =k\frac{q_1 q_2}{r^2}
   	\]
   	Suppose both $q_1$ and $q_2$ are positive, which means the two particles are being pushed away from each other (``opposites attract, likes repel''). Use integrals to answer the following.
    \begin{enumerate}
        \item The two particles are far apart. You now want to move them to be just $1$ meter apart. Put an upper bound on the amount of work required for this. Note that the work is done \textit{against} the force field and therefore is given by the integral of $-F(r)$ over distance from $r=a$ to $r=b$.%(work done against a force over a distance is $\int_a^b F(r) \,  dr$)
        \item Is it possible to move the particles so that they are $0$ meters apart?
    \end{enumerate}

\item The usual definition of the Gamma function $\Gamma(x)$ is in terms of an improper integral. See Tutorial 4 question 4 for a proof that $\Gamma(x) = x\Gamma(x-1)$ and $\Gamma(1) = 1$. For integers $n$, $(n+1)! = \Gamma(n)$. See the Gamma function below: 

\[
        \Gamma(x) = \int_0^\infty e^{-t}t^{x-1}\,\mathrm dt
    \]

The topic of this question is to prove that this integral converges for any $x\geq0$. 

\begin{enumerate}
    \item What happens when $x=1$?
    \item Investigate the integrand $e^{-t}t^{x-1}$. How do the values change as $x$ changes?

    \item We know that $\Gamma(n) = (n+1)!$. Use this fact and the comparison test to show that $\Gamma(x)$ converges when $x\geq1$.
    
    \item Let's show that $\Gamma(x)$ converges for $0<x<1$. Split up $\Gamma(x)$ into the following two integrals:

    \[\Gamma(x) = \Gamma_1(x) + \Gamma_2(x) = \int_0^1 e^{-t}t^{x-1}\mathrm dt + \int_1^\infty e^{-t}t^{x-1}\mathrm dt\]

    If both $\Gamma_1$ and $\Gamma_2$ converge, then we're good!

    \begin{enumerate}
        \item Use the comparison test to show $\Gamma_2(x)$ converges.

        \item Use the $p$-test and comparison test to show that $\Gamma_1(x)$ converges. 
    \end{enumerate}

    
\end{enumerate}



\end{enumerate}

%%%%%%%%%%%%%%%%%%%%%%%%%%