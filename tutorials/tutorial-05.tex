\documentclass[red]{tutorial}
\usepackage[no-math]{fontspec}
\usepackage{xpatch}
	\renewcommand{\ttdefault}{ul9}
	\xpatchcmd{\ttfamily}{\selectfont}{\fontencoding{T1}\selectfont}{}{}
	\DeclareTextCommand{\nobreakspace}{T1}{\leavevmode\nobreak\ }
\usepackage{polyglossia} % English please
	\setdefaultlanguage[variant=us]{english}
%\usepackage[charter,cal=cmcal]{mathdesign} %different font
%\usepackage{avant}
\usepackage{microtype} % Less badboxes

%\usepackage{enumitem}

\usepackage[charter,cal=cmcal]{mathdesign} %different font
%\usepackage{euler}
 
\usepackage{blindtext}
\usepackage{calc, ifthen, xparse, xspace}
\usepackage{makeidx}
\usepackage[hidelinks, urlcolor=blue]{hyperref}   % Internal hyperlinks
\usepackage{mathtools} % replaces amsmath
\usepackage{bbm} %lower case blackboard font
\usepackage{amsthm, bm}
\usepackage{thmtools} % be able to repeat a theorem
\usepackage{thm-restate}
\usepackage{graphicx}
\usepackage[dvipsnames]{xcolor}
\usepackage{multicol}
\usepackage{fnpct} % fancy footnote spacing
\usepackage{tikz}
\usetikzlibrary{arrows.meta}

\usepackage{pgfplots}
\pgfplotsset{compat=1.18}
%\pgfkeys{/pgf/fpu}

 
\newcommand{\xh}{{{\mathbf e}_1}}
\newcommand{\yh}{{{\mathbf e}_2}}
\newcommand{\zh}{{{\mathbf e}_3}}
\newcommand{\R}{\mathbb{R}}
\newcommand{\Z}{\mathbb{Z}}
\newcommand{\N}{\mathbb{N}}
\newcommand{\proj}{\mathrm{proj}}
\newcommand{\Proj}{\mathrm{proj}}
\newcommand{\Perp}{\mathrm{perp}}
\renewcommand{\span}{\mathrm{span}\,}
\newcommand{\Span}{\mathrm{span}\,}
\newcommand{\Img}{\mathrm{img}\,}
\newcommand{\Null}{\mathrm{null}\,}
\newcommand{\Range}{\mathrm{range}\,}
\newcommand{\rref}{\mathrm{rref}}
\newcommand{\rank}{\mathrm{rank}}
\newcommand{\Rank}{\mathrm{rank}}
\newcommand{\nnul}{\mathrm{nullity}}
\newcommand{\mat}[1]{\begin{bmatrix}#1\end{bmatrix}}
\newcommand{\chr}{\mathrm{char}}
\renewcommand{\d}{\mathrm{d}}


\theoremstyle{definition}
\newtheorem{example}{Example}[section]
\newtheorem{defn}{Definition}[section]

%\theoremstyle{theorem}
\newtheorem{thm}{Theorem}[section]

\pgfkeys{/tutorial,
	name={Tutorial 5},
	author={},
	course={MAT 187},
	date={},
	term={},
	title={Improper Integrals}
	}

\begin{document}
	\begin{tutorial}
		\begin{objectives}
	In this tutorial you will practice using the standard MAT187 integration techniques.
\end{objectives}

\vspace{-.5em}
\subsection*{Problems}
\vspace{-.5em}

%%%%%%%%%%%%%%%%%%%%%%%%%%

\begin{enumerate}
	\item 
	
\begin{enumerate}
    \item Find an anti-derivative of  $\frac{1}{x^2 - 9}$ using partial fraction decomposition.
    \item Find an anti-derivative of  $\frac{1}{9 + x^2}$ using trig substitution.
    \item (\textbf{Challenge}) Find an anti-derivative of  $\frac{1}{9 + x^2}$ using partial fraction decomposition. Does your answer match the answer you got using trig substitution?

    \emph{Hint:} You may need to use complex numbers during your partial fraction decomposition step, but after simplifying, you should be left with a real answer.

    If a complex number $z=re^{i\theta}$, then $\ln z = \ln r + i\theta$.
\end{enumerate}

\item The magnitude of the force acting on a particle of charge $q_1$ that is $r$ meters away from another particle of charge $q_2$ is
   	\[
	  	F =k\frac{q_1 q_2}{r^2}
   	\]
   	Suppose both $q_1$ and $q_2$ are positive, which means the two particles are being pushed away from each other (``opposites attract, likes repel''). Use integrals to answer the following.
    \begin{enumerate}
        \item The electrons start out at a distance of $10$m.
        The charge of the electrons is $q_1=q_2=1.6\cdot 10^{-19}$.

        How much work is required to move the electrons so that they
        are only $1m$ apart?
        
        \emph{Hint:} The work is done \textit{against} the force field and therefore is given by the integral of $-F(r)$ over distance from $r=a$ to $r=b$.%(work done against a force over a distance is $\int_a^b F(r) \,  dr$)
        \item Is it possible to move the electrons so that they are $0$ meters apart?
        \item If the electrons started very, very far away, would it be possible to bring them into a distance of $1m$ from each other with a 
        finite amount of work?
    \end{enumerate}

\item The usual definition of the Gamma function $\Gamma(x)$ is in terms of an improper integral: 
\[
        \Gamma(x) = \int_0^\infty e^{-t}t^{x-1}\,\mathrm dt.
    \]

In the last tutorial, you showed that $\Gamma(n)=n\Gamma(n-1)$, which means if $\Gamma(1)=1$, the function $\Gamma(x)$ will have a finite value for integers $x\geq 1$.


The topic of this question is to prove that this integral converges for any $x\geq0$. 

\begin{enumerate}
    \item Directly compute $\Gamma(1)$ from the integral definition of $\Gamma$.

    \item Consider the integrand $g_x(t)=e^{-t}t^{x-1}$.
    
    \begin{enumerate}
        \item For which values of $x$ does $g_x$ have a vertical asymptote? For which values of $x$ does $g_x$ have a horizontal asymptote?
        For which values of $x$ is $g_x$ a non-negative function?
        \item Show that if $t\geq 1$, then $g_{x_1}(t) \geq g_{x_2}(t)$ so
        long as $x_1\geq x_2\geq 1$.
    \end{enumerate}

    \item We know that $\Gamma(n) = (n+1)!$, and so $\Gamma$ at any integer is
     \emph{finite}. Using this fact, apply a comparison test to show that
      $\Gamma(x)<\infty$ when $x\geq1$.

      \emph{Hint:} You may need to do separate comparisons for $t<1$ and $t\geq 1$.
    
    \item Let's show that $\Gamma(x)$ converges for $0<x<1$. Split up $\Gamma(x)$ into the following two integrals:

    \[\Gamma(x) = \Gamma_1(x) + \Gamma_2(x) = \int_0^1 e^{-t}t^{x-1}\mathrm dt + \int_1^\infty e^{-t}t^{x-1}\mathrm dt\]

    If both $\Gamma_1$ and $\Gamma_2$ converge, then we're good!

    \begin{enumerate}
        \item Show that $e^{-t}t^{x-1} \leq t^{x-1}$ and then 
        use a comparison test to show that $\Gamma_1(x)$ converges. 
        \item Show that $e^{-t}t^{x-1} \leq e^{-t}$ and then use a comparison test to show $\Gamma_2(x)$ converges.
    \end{enumerate}

    
\end{enumerate}



\end{enumerate}

%%%%%%%%%%%%%%%%%%%%%%%%%%
	\end{tutorial}

	\begin{solutions}
		
	
\begin{enumerate}
\item 
\begin{enumerate}
    \item Evaluate $\int \frac{1}{x^2 - 9}\,\mathrm dx$ for $x > 3$ using partial fractions.	
	The partial fractions solution is relatively straightforward where the integrand can be written as $\frac{1}{x^2 - 9} = \frac{A}{x-3} + \frac{B}{x+3}$ for some constants $A$ and $B$. The constants are given by $A = \frac{1}{6}$ and $B = -\frac{1}{6}$. The integral then becomes:
	    
    \[
        \int\frac{\mathrm dx}{x^2-9} \\
        = \int\frac{1/6}{x-3} - \frac{1/6}{x+3}\mathrm dx \\
        = \frac{1}{6}\ln|x-3| - \frac{1}{6}\ln|x+3| + C
    \]
    
    Since $x > 3$, we can drop the absolute value signs, and after some simplification, the expression simplifies to $\frac{1}{6}\ln\left(\frac{x-3}{x+3}\right) + C$.

    \item We can use the trigonometric substitution $x = 3\tan{\theta} \implies \mathrm dx = 3\sec^2\theta \mathrm d\theta$.

    Then we can write
    \[
    \int \frac{3\sec^2\theta\mathrm d\theta}{9\tan^2\theta + 9}\\
    =\int\frac{\sec^2\theta\mathrm d\theta}{3\sec^2\theta} = \frac{1}{3}\int\mathrm d\theta = \frac{1}{3}\theta+C = \frac{1}{3}\tan^{-1}\left(\frac{x}{3}\right)+C
    \]

    \item We can write the integrand as: 
    \[
    \frac{1}{(x+3i)(x-3i)} = \frac{i/6}{x+3i} - \frac{i/6}{x-3i}
    \]

    Then we can write this indefinite integral as:
    \[
    \frac{i}{6}\ln(x+3i) - \frac{i}{6}\ln(x-3i) + C = \frac{i}{6}\ln\left(\frac{x+3i}{x-3i}\right) + C
    \]

    Using the provided formula, we can rewrite the natural logarithm expression as:

    \[
    \ln1 + 2i\tan^{-1}\left(\frac{3}{x}\right) = 2i\tan^{-1}\left(\frac{3}{x}\right) = 2i\left[\frac{\pi}{2} - \tan^{-1}\left(\frac{x}{3}\right)\right]
    \]

    where the last inequality can be shown by drawing a triangle. Then the indefinite integral can be written as:
    \[
    \int\frac{\mathrm dx}{9+x^2} = \frac{1}{3}\tan^{-1}\left(\frac{x}{3}\right) - \frac{\pi}{6} + C = \frac{1}{3}\tan^{-1}\left(\frac{x}{3}\right) + D
    \]

    where we let another constant $D = - \frac{\pi}{6} + C$. This expression is equivalent to the one from Part B.
\end{enumerate}
\item 
\begin{enumerate}
    \item The work is given by:
    \[
    \displaystyle\int_{10}^1 -\frac{\left(8.99\times10^9\right)\left(1.6\times10^{-19}\right)^2}{r^2}dr = 2.30\times10^{-28}
    \]
    \item 
    No, this would require infinite energy. For any $D>0$
    $$\int_D^0 -k\frac{q_1 q_2}{r^2} \, dr = \underset{t\rightarrow 0}{\lim} \Big( k\frac{q_1 q_2}{t} -  k\frac{q_1 q_2}{D}\Big) = \infty$$
    \item Yes, this is given by the following improper integral which has a finite value:
    $$\int_\infty^1 -k\frac{q_1 q_2}{r^2} \, dr = \underset{t\rightarrow \infty}{\lim} \Big( k\frac{q_1 q_2}{1} -  k\frac{q_1 q_2}{t}\Big) = kq_1q_2$$
\end{enumerate}
\item 
\begin{enumerate}
    \item Open up Desmos!
    \item We would like to apply integration by parts. Recall the integration by parts formula:

    \[
    \int_0^\infty u\mathrm dv = uv|_0^\infty \int_0^\infty \mathrm du
    \]

    To the integral:
    \[
        \gamma(n) = \int_0^\infty e^{-t}t^{n-1}\,\mathrm dt.
    \]    
    Let $dv = e^{-t}\mathrm dt$ and $u = t^{n-1}$. Then $\mathrm d u = (n-1)t^{n-2} \mathrm d t$ and $v = -e^{-t}$. So:

    
    \[
        \int_0^\infty e^{-t}t^{n-1}\,\mathrm dt = -e^{-t}t^{n-1}|_0^\infty -\int_0^\infty -e^{-t}(n-1)t^{n-2}dt
    \]
    
    Since $-e^{-t}t^{n-1}|_0^\infty = 0$, we can see:
    \[
        =(n-1)\int_0^\infty e^{-t}t^{n-2}\mathrm dt =(n-1) \Gamma(n-1) 
    \]

    So:
    \[
        \Gamma(n) = (n-1)\Gamma(n-1) 
    \]

    \item We know that $-e^{-t}$ is an antiderivative for $e^{-t}$. So:

    \[
        \Gamma(1) = \int_0^\infty e^{-t} \mathrm dt
    \]


    
    \item In this case, $ \Gamma(4) = 3\Gamma(3) = 3\cdot 2 \Gamma(2) = 3\cdot 2 \cdot 1 \Gamma(1) = 3!$. 

    \item For general $n$, $\Gamma(n) = (n-1)!$.

    \item The recursive formula $\Gamma(x) = (x-1)\Gamma(x-1)$ will work for general $x > 1$, since we never used the fact that $n$ was an integer during integration by parts, only that $n>1$. It's only important for $n$ to be an integer so that we can relate $\gamma(n)$ to $\gamma(1)$. 

    
\end{enumerate}
\item 
    \begin{enumerate}
        \item 
        \begin{enumerate}
            \item For $x<1$, $g_x$ has a vertical asymptote. $g_x$ has a horizontal asymptote regardless of what $x$ is, and it is always nonnegative over the region $(0,\infty)$ that we care about for this question.
            \item  Since $x_1 \geq x_2\geq 1$, it follows $t^{x_1} \geq t^{x_2}$. Because all other factors are positive for $t>0$, $e^{-t}t^{x_1-1} \geq e^{-t}t^{x_2-1}$.
        \end{enumerate}
        \item For this part we assume that $x\geq 1$. Using the previous part, we can see that $g_n\geq g_x$ provided $x\leq n$. Since $g_x$ is positive by the comparison test, we can see that the following integral converges:

        \[
            \int_1^\infty g_x(t) \mathrm dt \leq \int_1^\infty g_n(t) <\infty 
        \]

        On the other hand, if $0< t \leq 1$, then $t^{x-1} \leq 1$, and $t^{x-1}e^{-t} \leq e^{-t}$. So we can apply the comparison test:

        \[
            \int_0^1 e^{-t}t^{x-1} \leq \int_0^1 e^{-t} = e^{-1} - 1
        \]
        So the integral converges. Since both these integrals converge, we know that the full integral converges, provided $x\geq 1$: 

        \[
           \int_0^\infty e^{-t} t^{x-1} \mathrm dt <\infty  
        \]
        
        \item Now we suppose that $0< x \leq 1$. 

        \begin{enumerate}
            \item Suppose that $0\leq t \le q1$. Then $e^{-t}\leq 1$, so the integrand $e^{-t}t^{x-1} \leq t^{x-1}$. So:

            \[
                \int_0^1 e^{-t}t^{x-1} \mathrm dt \leq \int_0^1 t^{x-1} \mathrm dt 
            \]

            Since $x>0$, $x-1>-1$. By the p-test, the integral on the right converges!

            \item Suppose that $1 \leq t \leq \infty$. Then $t^{x-1} \leq 1$. So $e^{-t}t^{x-1} \leq e^{-t}$. By the comparison test:

            \[
            \int_1^\infty e^{-t} t^{x-1} \leq \int_1^\infty e^{-t} = -e^{-t}|_0^\infty = 1
            \]
        \end{enumerate}
        Since both of these integrals converge, the full integral $\int_0^\infty e^{-t}t^{x-1}\mathrm dt$ converges! 
        
    \end{enumerate}
\end{enumerate}
	
	\end{solutions}
	\begin{instructions}
		\subsection*{Learning Objectives}
Students need to be able to\ldots
\begin{itemize}
	\item Evaluate improper integrals.
\end{itemize}

\subsection*{Notes}
	\begin{enumerate}
			\item The first two parts should be fairly standard by applying the standard trigonometric substitution and partial fraction techniques. The third part can be quite challenging since it involves manipulating the inverse tangent function using triangles and realizing that the constant of integration can include a separate term.

            \item This question should be straightforward if the students follow the instructions in the questions. They may need some guidance on what it means for the electrons to be far apart.

            \item This question mainly involves applying integration by parts once - the difficult part is the abstraction, with multiple variables involved. Clarifying which variables are constant with respect to other operations is important, and they may need some help with that. 

            \item The students dont have to get through this question. 
	\end{enumerate}
	\end{instructions}

\end{document}
