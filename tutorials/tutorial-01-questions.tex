		\begin{objectives}
			In this tutorial, you practice creating and using polynomial approximations of functions.
		\end{objectives}

	%	\bigskip


		\subsection*{Problems}
		
		\begin{enumerate}
			\item  The inverse square root function, $f(x) = 1/\sqrt{x}$ is an important function
			when simulating light and reflections in computer graphics.  However, computing square 
			roots can be expensive, so we will use much-simpler polynomial approximations.
				\begin{enumerate}
					\item Find the first and second degree Taylor approximations to $f$ at $x=4$.
					Use each of these polynomials to approximate $f(2)$. Use a calculator 
					to quantify the error in each approximation.
					\item Find the quadratic polynomial that interpolates between the points 
					$(3,f(3))$, $(4,f(4))$, and $(5,f(5))$. 
					Use this polynomial to approximate $f(2)$ and quantify its error.

					\item Use Desmos to graphically answer the following: 
					In what region(s) is the interpolating polynomial better?
					In what region(s) is the Taylor polynomial better?

					\item Find the third and fourth degree Taylor approximations to $f$ at $x=4$.
					How well do they approximate $f(0.5)$. What about $f(7.5)$?

					\item Write down the remainder formula for the $n$th order 
                    Taylor approximation of $f$. Can you use this formula to interpret the error as $x$ tends to zero, and as $x$ moves away from $4$?
				\end{enumerate}

				\item The step function $H(x)$  defined by 
                
                \[H(x) = \begin{cases}
				    0  \text{ if } x < 0\\
                        1 \text { if } x \geq 1
				\end{cases}\] 
                
                is an important function in circuits and control theory, and is used to distinguish between an on and off state. For example, $H(x^2-1)$ indicates the region where $x^2 \geq 1$ holds true. Approximating $H(x)$ by ``smoother'' functions enables the use of calculus, and has an important application in machine learning. 
                \begin{enumerate}
                    \item Approximate by Taylor series
                    \begin{enumerate}
                        \item The logistic function $L_k(x) = \frac{1}{1+e^{-2kx}}$ with scaling $k$ is a good approximation for the step function as $k$ tends to infinity. Find a third order Taylor polynomial $p(x)$ for $L_k(x)$ based at $x=0$.
                        \item What is the largest error between $L_k(x)$ and its Taylor polynomial over the region $[-1,1]$?  
                        \item Let $\epsilon >0$ be small. What is the largest error between $L_k(x)$ and $H(x)$ over the region $[\epsilon,1]$? The answer will be a function of $\epsilon$ and $k$. What happens to the error when $k\to \infty$?
                        
                        \item Estimate the error over the interval $|x|\leq 1, |x| \geq \epsilon$ when approximating $H(x)$ by $p(x)$. 

                    \end{enumerate}
                
                
                \item Approximation by Polynomial Interpolation

                \begin{enumerate}
                    \item Pick four interpolation points $x_1,x_2,x_3,x_4 \in [-1,1]$, and construct a third order polynomial $p(x) = ax^3 + bx^2 +cx + d$ interpolating $H(x)$.   
                    \item Observe the error $|H(x) - p(x)|$ over the interval $[-1,1]$.  
                \end{enumerate}
                
                \end{enumerate} 

				\item Consider the functions $e^x$, $\cos{x}$, $\sin{x}$, $\cosh{x}$, and $\sinh{x}$.
                \begin{enumerate}
                    \item Write the 4th order Taylor polynomials for the above functions centred around $x=0$
                    \item Write the Taylor polynomials for $e^{ix}$ and $\cos{x} + i\sin{x}$, also centred around $x=0$. Which identity does this remind you of?
                    \item Write the 4th order Taylor polynomial for $\cosh{x} + \sinh{x}$ centred around $x=0$. Which identity does this remind you of?
                \end{enumerate}
		\end{enumerate}
